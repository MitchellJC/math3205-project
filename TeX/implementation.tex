% Specs and general overview
Models were run on a Windows 11 computer with a AMD Ryzen 5 5625U 2.3GHz processor with 16 GB of RAM\@. Gurobi version 10.0.1 was used with Python version 3.9.16. We implemented two different models for LBBD, one following\cite{roshanaei2017propagating} which solves the master problem iteratively in a loop, rebuilding the master problem branch-and-bound tree on every iteration. The other model leveraged lazy constraints in Gurobi to perform a branch and check routine\cite{LBBDBible}, building the branch-and-bound tree once while applying Benders' cuts as lazy constraints at each new incumbent solution. The pure MIP and network model were solved using standard Gurobi routines. 

% Implementation difficulty
Model formulations were simple enough to interpret but practical implementation was difficult for a few reasons. Firstly, the original paper did not supply any code. This made it hard to find implementation details that may have drastically improved results, for example some important caching protocol may have been used that was not mentioned in the paper. Further, the paper supplied distribution parameters that were useful in generating data, however it did not supply the data instances that were used to generate their results. This made it impossible to verify if the differences seen in our results were due to an implementation error, random chance or something else. It also left data inadequately described. For example the surgery times were said to be from a normal distribution, this might imply that they should be real-valued. However, upon inspection of another paper's data which worked on a similar problem\cite{guo}, it was noticed that surgery times were rounded to some integer value. Implementing this with our data generation scheme immediately improved the performance of our models. We reached out to the author of the original paper and we were told that they were no longer in possession of either the code or data that was used. 

The paper did not supply example objective values found for their instances which made it hard to determine feasibility of our solutions. Ultimately the main verification method of our results was by the consistency of outputted optimal objective values across all models implemented. That is, the pure MIP, Network and Benders' models were implemented in sufficiently different ways but all provided the same output giving us some level of confidence in our implementation's optimality. Feasibility was verified by manual inspection of patient allocations for instances with small number of patients. We had no way of verifying if the speed of our models would be the same as achieved by the models used in the original paper.