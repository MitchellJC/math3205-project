% Runtime specifications
Models were run on a Windows 11 computer with a AMD Ryzen 5 5625U 2.3GHz processor with 16 GB of RAM\@. Gurobi version 10.0.1 was used with Python version 3.9.16. 

We aimed to compare 7 models: The pure MIP, the network model, an iterative LBBD1 (iLBBD1), LBBD1 using lazy constraints (cLBBD1), iterative LBBD2 with propagation (iLBBD2p), LBBD2 with lazy constraints and propagation (cLBBD2p) and LBBD4 with lazy constraints and propagation (cLBBD4). The pure MIP was used as a baseline.\ iLBBD1 and cLBBD1  were chosen to be used as baseline LBBD models to compare lazy constraints to iteration. LBBD2 with propagation was chosen as it was one of the best performers in th original papers, with the idea to use it as the main comparison to the network model and our new cut. We chose to collect results for LBBD4 only using propagation and lazy constraints as we believed this would give us the best results while adhering to the project time constraints.

Unlike in the original paper, where models were ran with a time limit of 7200 seconds, we set our time limit to 900 seconds due to project time constraints. For this same reason we only ran models with 5 available operating rooms which should lead to harder sub problems and relatively easier master problems\cite{roshanaei2017propagating}. We run the models on 5 different instances of data for each number of patients. There was uncertainty as to whether the models were run to true optimality or to a relative MIP gap of $1\%$ as was done by Guo\cite{guo}, so we report results for both scenarios. We follow~\cite{roshanaei2017propagating} by defining the best performing model to be the most robust, that is, the one able to solve the model to optimality within the given time constraints.

\begin{table*}
    \centering
    \caption{Average time (seconds) until solved to optimality over 5 instances. The number of instances not solved to optimality are superscripted. Non-solved instances are not included in average. **** represents that no instances solved in time.}
    \begin{tabular}{rrrrrrrr} \toprule
        $|\mathcal{P}|$ & Pure MIP & Network & iLBBD1 & cLBBD1 & iLBBD2p & cLBBD2p & cLBBD4p \\ \midrule
        20              & 16.06 &     ${****}^{(5)}$    & 1.509 &  0.8829 & 1.431 & 0.8800 & 0.7890 \\
        40              & $179.3^{(2)}$ & $268.5^{(3)}$   &  $4.429$ & $1.911^{(4)}$ & $4.503^{(1)}$ & $1.911^{(4)}$ & $1.959^{(4)}$ \\
        60 & $30.80^{(4)}$ & ${****}^{(5)}$ & $24.46^{(1)}$ & $10.73^{(4)}$ & $34.61^{(2)}$ & $21.46^{(4)}$ & $25.80^{(4)}$ \\
        80 & ${****}^{(5)}$ & ${****}^{(5)}$ & ${****}^{(5)}$ & ${****}^{(5)}$ & ${****}^{(5)}$ & ${****}^{(5)}$ & ${****}^{(5)}$ \\
        \bottomrule
    \end{tabular}
\end{table*}


\begin{table*}
    \centering
    \caption{Average gap (\%) over 5 instances after trying to solve to optimality. MIPGap is reported for pure MIP, Network and callback implementations of LBBD.\@ Gap between master problem lowerbound and best sub problem upperbound is report for iterative implementations of LBBD.}\label{tab:avgGapOpt}
    \begin{tabular}{rrrrrrrr} \toprule
        $|\mathcal{P}|$ & Pure MIP & Network & iLBBD1 & cLBBD1 & iLBBD2p & cLBBD2p & cLBBD4p \\ \midrule
        20              & 0.000 &    $0.4950$     & 0.000 &  0.000 & 0.000 & 0.000 & 0.000 \\
        40              & $0.04220$ & $0.3382 $  & $0.6252 $ & $0.3532 $ & $0.09500 $ & $0.4765 $ & $0.4948$ \\
        60 & $0.1827 $ & $0.4131 $ & $0.5514 $ & $0.2540 $ & $0.1700 $ & $0.3513 $ & $0.3273 $ \\
        80 &  $0.2630 $ &  $0.2048 $ & $0.9390$ &  $0.6000 $ & $0.2100$ & $0.4718 $ & $0.5218$ \\
        \bottomrule
    \end{tabular}
\end{table*}
