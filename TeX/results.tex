% Runtime specifications
Models were run on a Windows 11 computer with a AMD Ryzen 5 5625U 2.3GHz processor with 16 GB of RAM\@. Gurobi version 10.0.1 was used with Python version 3.9.16. 

Unlike in the original paper, where models were ran with a time limit of 7200 seconds, we set our time limit to 900 seconds due to project time constraints. For this same reason we only ran models with 5 available operating rooms which should lead to harder sub problems and relatively easier master problems\cite{roshanaei2017propagating}. We aimed to compare 7 models: The pure MIP, the network model, an iterative LBBD1 (iLBBD1), LBBD1 using lazy constraints (cLBBD1), iterative LBBD2 with propagation (iLBBD2p), LBBD2 with lazy constraints and propagation (cLBBD2p) and LBBD4 with lazy constraints and propagation (cLBBD4). We run the models on 5 different instances of data for each number of patients. There was uncertainty as to whether the models were run to true optimality or to a relative MIP gap of $1\%$ as was done by Guo\cite{guo}, so we report results for both scenarios. We follow~\cite{roshanaei2017propagating} by defining the best performing model to be the most robust, that is, the one able to solve the model to optimality within the given time constraints.
% Should probably move next paragraph to discussion.
It should be emphasised that because our time constraints are were stricter robustness may have been greatly affected giving different results to the original paper, however we found that our results were already disparate without this factor. For example, the original paper was able to solve with patient sizes up to 100 without ever going over a 200 second run-time.

\begin{table*}
    \centering
    \caption{Average time (seconds) until solved to optimality over 5 instances. The number of instances not solved to optimality are superscripted. Non-solved instances are not included in average. **** represents that no instances solved in time.}
    \begin{tabular}{rrrrrrrr} \toprule
        $|\mathcal{P}|$ & Pure MIP & Network & iLBBD1 & cLBBD1 & iLBBD2p & cLBBD2p & cLBBD4p \\ \midrule
        20              & 16.06 &     ${****}^{(5)}$    & 1.509 &  0.8829 & 1.431 & 0.8800 & 0.7890 \\
        40              & $179.3^{(2)}$ & $268.5^{(3)}$   &  $4.429$ & $1.911^{(4)}$ & $4.503^{(1)}$ & $1.911^{(4)}$ & $1.959^{(4)}$ \\
        60 & $30.80^{(4)}$ & ${****}^{(5)}$ & $24.46^{(1)}$ & $10.73^{(4)}$ & $34.61^{(2)}$ & $21.46^{(4)}$ & $25.80^{(4)}$ \\
        80 & ${****}^{(5)}$ & ${****}^{(5)}$ & ${****}^{(5)}$ & ${****}^{(5)}$ & ${****}^{(5)}$ & ${****}^{(5)}$ & ${****}^{(5)}$ \\
        \bottomrule
    \end{tabular}
\end{table*}


\begin{table*}
    \centering
    \caption{Average gap (\%) over 5 instances after trying to solve to optimality. MIPGap is reported for pure MIP, Network and callback implementations of LBBD.\@ Gap between master problem lowerbound and best sub problem upperbound is report for iterative implementations of LBBD.}\label{tab:avgGapOpt}
    \begin{tabular}{rrrrrrrr} \toprule
        $|\mathcal{P}|$ & Pure MIP & Network & iLBBD1 & cLBBD1 & iLBBD2p & cLBBD2p & cLBBD4p \\ \midrule
        20              & 0.000 &    $0.4950$     & 0.000 &  0.000 & 0.000 & 0.000 & 0.000 \\
        40              & $0.04220$ & $0.3382 $  & $0.6252 $ & $0.3532 $ & $0.09500 $ & $0.4765 $ & $0.4948$ \\
        60 & $0.1827 $ & $0.4131 $ & $0.5514 $ & $0.2540 $ & $0.1700 $ & $0.3513 $ & $0.3273 $ \\
        80 &  $0.2630 $ &  $0.2048 $ & $0.9390$ &  $0.6000 $ & $0.2100$ & $0.4718 $ & $0.5218$ \\
        \bottomrule
    \end{tabular}
\end{table*}
