\subsection{Master Problem} 

The following is the LBBD formulation of the problem.
For a pure IP formulation refer to \cite{roshanaei2017propagating}. We will first outline
the formulation of the master problem. The master problem handles assignment of patients
to hospital-days.

\begin{table}[H]
% \captionsetup{font={Large,sf}}
\caption*{\bf{SETS}}
\resizebox{0.45\textwidth}{!}{
\begin{tabular}{ll}
    $\mathcal{P}$ &  Set of patients $p \in \mathcal{P}$\\
    $\mathcal{P}{'} $ & Set of mandatory patients, $\mathcal{P}{'} = \lbrace p|\rho_{p}(|\mathcal{D}|-\alpha_{p})\leq-\Gamma\rbrace$\\
    $\mathcal{H} $ & Set of hospitals, $h \in \mathcal{H}$\\
    $\mathcal{D} $ & Set of days in the planning horizon, $d \in \mathcal{D}$\\
    $\mathcal{R}_{h} $ & Set of ORs in each hospital's surgical suite, $r \in \mathcal{R}_{h}$\\
\end{tabular}
}

\end{table}

\begin{table}[H]
\caption*{\bf{DATA}}
\label{tab:MPdata}
\resizebox{0.45\textwidth}{!}{
\begin{tabular}{ll}
    $G_{hd}$ & Cost of opening the surgical suite in hosptial h on\\
    &  $\quad$ day d.\\
    $F_{hd} $ & Cost of opening and OR in hospital h on day d. \\
    $B_{hd} $ & Regular operating hours of each OR on day d. \\
                & $\quad$ in hospital h.\\
    $T_{hp} $ & Total booked time (preparation time + surgery time\\
               & $\quad$ + cleaning time) of patient p.\\
    $\rho_{p} $ & Health status score assigned to patient p.\\
    $\alpha_{p} $ & Number of days elapsed from the referal date of \\
    & $\quad$ patient p.\\
    $\kappa_{1} $ & Waiting cost for scheduled patients.\\
    $\kappa_{2} $ & Waiting cost for unscheduled patients.\\
    $\Gamma $ & Health status threshold above which patients have to\\
    & $\quad$ be operated.
\end{tabular}
}
\end{table}
\begin{table}[H]
\caption*{\bf{VARIABLES}}
\label{tab:MPvariables}
\resizebox{0.45\textwidth}{!}{
\begin{tabular}{ll}
    $x_{hdp} $ & 1 if patient p is assigned to hospital h on day d\\
    & $\quad$, 0 otherwise\\
    $u_{hd} $ & 1 if the surgical suite in hospital h is opened\\
    & $\quad$ on day d, 0 otherwise\\
    $y_{hd} $ & $\in \mathbb{Z}^+$,  lower bound on number of operating rooms \\
    & $\quad$ open in hospital h on day d \\
    $w_{p}$ & 1 if patient p is not scheduled this horizon,\\
    & $\quad$ 0 otherwise\\ 
\end{tabular}
}
\label{MP:variables}
\end{table}
\leavevmode
% \newline
% %setting up the master problem
\subsubsection*{Objective}

The objective function balances the minimisation of costs associated with opening hospitals
 and ORs and maximising the reward of assigning patients to surgeries. 
\begin{align}
\operatorname{minimize} \bigg( &\sum\limits_{h \in \mathcal{H}} \sum\limits_{d \in \mathcal{D}} G_{hd} U_{hd} 
 + \sum\limits_{h \in \mathcal{H}}\sum\limits_{d \in \mathcal{D}} F_{hd} y_{hd} \label{eq:MPobjective}
\\ &+ \sum\limits_{h \in \mathcal{H}} \sum\limits_{d \in \mathcal{D}} \sum\limits_{p \in \mathcal{P}}  
\kappa_{1} [\rho_{p} (d - \alpha_{p}) x_{hdp}]\notag
\\ &+ \sum\limits_{p \in \mathcal{P} \setminus \lbrace \mathcal{P}{'} \rbrace } \kappa_{2} [\rho_{p}( \mathcal{D} + 1 -\alpha_{p} ) w_{p}]
\bigg) \notag
\end{align}

\subsubsection*{Constraints} The constraints for the MP are formulated as follows.
\begin{align}
    \sum\limits_{h \in \mathcal{H}} \sum\limits_{d \in \mathcal{D}}x_{hdp} = 1 
        && \forall p \in \mathcal{P}' \label{MPcon1}\\
    \sum\limits_{h \in athcal{H}} \sum\limits_{d \in \mathcal{D}} x_{hdp} + w_p = 1
        && \forall p \in \mathcal{P} \backslash \{\mathcal{P}'\}\label{MPcon2}\\
    x_{hdp} \leq u_{hd}
        && \forall h \in \mathcal{H}, d \in \mathcal{D}, p \in \mathcal{P}\label{MPcon3}\\
    \sum\limits_{p \in \mathcal{P}}T_px_{hdp} \leq |\mathcal{R}_h|B_{hd}u_{hd}
        && \forall h \in \mathcal{H}, d \in \mathcal{D} \label{MPcon4}\\
    T_p\,x_{hdp} \leq B_{hd} 
        && \forall h \in \mathcal{H}, d \in \mathcal{D}, p \in \mathcal{P}\label{MPcon5}\\
     \frac{\sum_{p \in \mathcal{P}}T_px_{hdp}}{B_{hd}} \leq y_{hd}
        && \forall h \in \mathcal{H}, d\in \mathcal{D} \label{MPcon6}\\
        y_{hd} \leq |\mathcal{R}_h|
        && \forall h\in\mathcal{H}, d \in \mathcal{D} \label{MPcon7}\\
        u_{hd}, x_{hdp} \in \{ 0,1\}
        && \forall h\in \mathcal{H}, d \in \mathcal{D}, p \in \mathcal{P}\label{MPcon8}\\
        w_p \in \{0,1\}
        && \forall p \in \mathcal{P} \backslash\{\mathcal{P}'\}\label{MPcon9}
\end{align}
Constraint (\ref{MPcon1}) ensures all mandatory patients are assigned in the planning 
horizon. Constraint (\ref{MPcon2}) ensures that variables $x_{hdp}$ and $u_{hd}$ are not
turned on simultaneously. Constraint (\ref{MPcon3}) ensures that if a patient is assigned
a hospital-day then that hospital-day is open. (\ref*{MPcon4}) ensures that surgery time 
of patients assigned to a hospital-day does not exceed the available surgery time in
that hospital day. (\ref*{MPcon5}) ensures an individuals surgery time does not exceed 
the hospital-day's available time. (\ref*{MPcon6}) ensures $y_{hd}$ gives a lower bound 
on the number of operating rooms. (\ref*{MPcon7}) ensures $y_{hd}$ does not exceed the
number of operating rooms available on a hospital-day. Constraints (\ref*{MPcon8}) -- (\ref*{MPcon9}) simply restrict
variables to binary.

\subsection{Subproblems}
Given a solution to the master problem $(\widehat{Y}^{(i)}_{hd}, \widehat{\mathcal{P}}^{(i)}_{hd})$ the 
subproblem minimises the number of ORs to open for a given hospital day. Each subproblem is
formulated as follows.

\begin{table}[H]
    \caption*{\bf{ADDITIONAL VARIABLES}}
    \resizebox{0.45\textwidth}{!}{
        \begin{tabular}{ll}
            $y_r$ & $\in \mathbb{Z}^+$, number of open operating rooms. \\
            $x_{pr}$ & 1 if patient p is assigned to operating room r,\\
            &   $\quad$ 0 otherwise.
        \end{tabular}
    }
\end{table}
\begin{align}
    \operatorname*{minimise} \quad \overline{Y}_{hd} = \sum\limits_{r \in \mathcal{R}_h}y_r
\end{align}

With constraints given as follows:
\begin{align}
    \sum\limits_{r\in\mathcal{R}_h}x_{pr} = 1 && \forall p \in \widehat{\mathcal{P}}^{(i)}_{hd}\label{SPcon1}\\
    \sum\limits_{p \in \widehat{\mathcal{P}}^{(i)}_{hd}} T_px_{pr} \leq B_{hd}y_r 
    && \forall r \in \mathcal{R}_h \label{SPcon2}\\
    x_{pr} \leq y_r && \forall p \in \widehat{\mathcal{P}}^{(i)}_{hd}, r \in \mathcal{R}_h \label{SPcon3}\\
    y_r \leq y_{r-1} && \forall r \in \mathcal{R}_h \backslash \{1\}\label{SPcon4}\\
    x_{pr},\,y_r\in\{0,1\} && \forall p \in \widehat{\mathcal{P}}^{(i)}_{hd}, r \in \mathcal{R}_h\label{SPcon5}
\end{align}
Constraint (\ref*{SPcon1}) ensures that each patient is assigned to only one operating room.
Constraint (\ref*{SPcon2}) ensures that no OR is overcapacitated. Constraint (\ref*{SPcon3})
ensures that patients are assigned to open ORs. Constraint (\ref*{SPcon4}) breaks symmtry
among ORs.

\subsection{Benders Cuts}
There are multiple forms of benders cuts outlined in the original paper. We will discuss
pertinent forms based on their performance as discussed in the original paper. These are 
the LBBD1 and LBBD2 benders cuts. In order to describe these cut types, we will first discuss the 
first-fit decreasing heuristic algorithm (FFD) as this is used to determine optimality of SPs. 

\subsubsection*{First-fit decreasing heuristic algorithm}
Since the SP packing problem can be difficult to solve, we can first find a feasible solution 
($\overline{F}^{(i)}_{hd}$) using a FFD heuristic. This process is faster than other techniques
such as integer and constraint programming. Moreover, we have the following relationship 
between the FFD, MP and SP solutions;
\begin{equation}
    \tilde{Y}^{(i)}_{hd} \leq \overline{Y}^{(i)}_{hd} \leq \overline{F}^{(i)}_{hd}
\end{equation}
We can use the FFD solution $\left(\overline{F}^{(i)}_{hd}\right)$ to find an optimal SP solution 
$\left(\overline{Y}^{(i)_{hd}}\right)$ without explicitly solving the SP. Moreover, if 
$\overline{F}^{(i)}_{hd} \neq \tilde{Y}^{(i)}_{hd}$ then when solving the SP we can use 
$\operatorname{min}\{\overline{F}^{(i)}_{hd},\, |\mathcal{R}_h|\}$ as an upper bound. 

\subsubsection{LBBD1}
LBBD1~\cite{roshanaei2017propagating} utilises both feasibility and optimality cuts to 
correct the MP to find a solution. If the SP is infeasible the following "no good" cut 
is added to the MP which requires at least one patient be removed from 
$\tilde{\mathcal{P}}^{(i)}_{hd}$.
\begin{align}
    \sum\limits_{p \in \tilde{\mathcal{P}}^{(i)}_{hd}}(1-x_{hdp})\geq 1 
        && \forall (h,d) \in \mathcal{U}_{hd}^{(i)}
\end{align}
Where $\mathcal{U}_{hd}^{(i)}$ is the set of infeasible SPs at this stage of solving the MP.
If the SP is optimal, that is $\tilde{Y}^{(i)}_{hd} = \overline{Y}^{(i)}_{hd}$, no cuts
are required. However, if the this is not the case, the following optimality cut is added:
\begin{align*}
    y_{hd} \geq \overline{Y}_{hd}^{(i)} - \sum\limits_{p \in \hat{\mathcal{P}}}(1-x_{hdp}) && \forall (h,d) \in \overline{\mathcal{J}}^{(i)}.
\end{align*}
Where $\overline{\mathcal{J}}^{(i)}$ is the set of SPs that are not optimal at a particular stage of solving the MP. This cut effectively has two results. It either forces at least one more OR open, or removes one patient from $\tilde{\mathcal{P}}^{(i)}_{hd}$.
\subsubsection{LBBD2}
LBBD2 differs from LBBD1 in its feasibility cut which is given as follows:
\begin{align*}
    y_{hd} \geq (|\mathcal{R}_h| + 1) - \sum\limits_{p \in \hat{\mathcal{P}}_{hd}^{(i)}}(1-x_{hdp}) && (h,d) \in \overline{\mathcal{U}}^{(i)}.
\end{align*}
If $\overline{Y}_{hd}^{(i)} = |\mathcal{R}_h|$ and the SP is infeasible then this cut simply removes one patient from $\overline{\mathcal{P}}_{hd}^{(i)}$. However, if $\hat{Y}_{hd}^{(i)} \le |\mathcal{R}_h|$ and the SP is infeasible then the cut either removes two patients from $\overline{\mathcal{P}}_{hd}^{(i)}$, or removes one patient and/or opens at least one more OR.
\subsubsection{Cut Propagation}
The original paper utilises cut propagation for LBBDs in order to generate multiple cuts for each infeasible SP. This is done by recognising that an infeasible set of patients for a particular hospital-day cannot be packed into a hospital-day with less or equal OR time.
\subsection{Network Problem}
We also give the formulation of the problem as a network. The problem contains the sets
and data of the master problem previously outlined with the addition of the following 
sets and data. Nodes represent hospital-day-time, where time is in minute intervals. Arcs represent assignment of patients to operating rooms.
\begin{table}[H]
    % \captionsetup{font={Large,sf}}
    \caption*{\bf{SETS}}
    \resizebox{0.4\textwidth}{!}{
    \begin{tabular}{ll}
        $\mathcal{N}$ &  Set of nodes $n \in \mathcal{N}$ for each hospital-day-minute\\
        $\mathcal{N^\prime}$ & Set of nodes $n \in \mathcal{N}$ \\
        & $\operatorname{s.t.} \operatorname{minDur} \leq n[time] \leq B_{n[time],n[day]} - \operatorname{minDur} $ \\
        $\mathcal{A}$ &  Set of arcs $a \in \mathcal{A}$ between nodes $n \in N$\\
    \end{tabular}
    }
    \end{table}
    \begin{table}[H]
        % \captionsetup{font={Large,sf}}
        \caption*{\bf{DATA}}
        \resizebox{0.45\textwidth}{!}{
        \begin{tabular}{ll}
            $t_n$ & Arcs that enter node $n \in \mathcal{N}$ \\
            $f_n$ & Arcs that leave node $n \in \mathcal{N}$\\
            $h_a$ & Hospital associated with arc $a \in \mathcal{A}$\\
            $d_a$ & Day associated with arc $a \in \mathcal{A}$ \\
            $\operatorname{patient}_a$ & Patient associated with arc $a \in \mathcal{A}$\\
            $\operatorname{start}_a$ & Start time of arc $a \in \mathcal{A}$\\
            $\operatorname{end}_a$ & End time of arc $a \in \mathcal{A}$\\
            $\operatorname{minDur}$ & minimum surgery duration 
        \end{tabular}
        }
        \end{table}
    The variables are also shared with the master problem with the addition of the following variable. 
\begin{table}[H]
    \caption*{\bf{VARIABLES}}
    \label{tab:NETvariables}
    \resizebox{0.4\textwidth}{!}{
    \begin{tabular}{ll}
        $z_{a}$ & 1 if arc a is turned on, 0 otherwise.
    \end{tabular}
    }
    \end{table}

    \subsubsection*{Objective}
    The objective function is identical to master problem objective given in \ref{eq:MPobjective}.
    \begin{align}
    \operatorname{minimize} \bigg( &\sum\limits_{h \in \mathcal{H}} \sum\limits_{d \in \mathcal{D}} G_{hd} U_{hd} 
     + \sum\limits_{h \in \mathcal{H}}\sum\limits_{d \in \mathcal{D}} F_{hd} y_{hd}
    \\ &+ \sum\limits_{h \in \mathcal{H}} \sum\limits_{d \in \mathcal{D}} \sum\limits_{p \in \mathcal{P}}  \label{MP:objective}
    \kappa_{1} [\rho_{p} (d - \alpha_{p}) x_{hdp}]
    \\ &+ \sum\limits_{p \in \mathcal{P} \setminus \lbrace \mathcal{P}{'} \rbrace } \kappa_{2} [\rho_{p}( \mathcal{D} + 1 -\alpha_{p} ) w_{p}]
    \bigg) \notag
    \end{align}
    
    \subsubsection*{Constraints} The constraints for the network formulation are given as follows,
    \begin{align}
        \sum\limits_{a \in f_n} z_a = \sum\limits_{a \in t_n} z_a 
            && \forall n \in \mathcal{N}^\prime \label{NETcon1:room_flow}\\
        \sum\limits_{\substack{a \in \mathcal{A} | \\ (h_a, d_a, \operatorname{start}_a) \\= (h,d,0)}}z_a \leq y_{hd}
            && \forall d \in \mathcal{D} \forall h \in \mathcal{H}\label{NETcon2:restrict_ops_by_ors}\\
        \sum\limits_{\substack{a \in \mathcal{A} |\\ (h_a,d_a,\operatorname{patient}_a)\\ =(h,d,p)}}z_a= x_{hdp}
            && \forall h \in \mathcal{H}, \forall d \in \mathcal{D}, \forall p \in \mathcal{P}  \label{NETcon3:is_patient_operated_on} \\
        \sum\limits_{h \in \mathcal{H}} \sum\limits_{d \in \mathcal{D}}x_{hdp} = 1 
            && \forall p \in \mathcal{P}' \label{NETcon4:must_do_mandatory}\\
        \sum\limits_{h \in \mathcal{H}} \sum\limits_{d \in \mathcal{D}} x_{hdp} + w_p = 1
            && \forall p \in \mathcal{P} \backslash \{\mathcal{P}'\}\label{NETcon5:turn_on_w}\\
        x_{hdp} \leq u_{hd}
            && \forall h \in \mathcal{H}, d \in \mathcal{D}, p \in \mathcal{P}\label{NETcon5:force_hosp_on}\\
            y_{hd} \leq |\mathcal{R}_h|
            && \forall h\in\mathcal{H}, d \in \mathcal{D} \label{NETcon6:max_OR}\\
            u_{hd}, x_{hdp} \in \{ 0,1\}
            && \forall h\in \mathcal{H}, d \in \mathcal{D}, p \in \mathcal{P}\label{con8}\\
            w_p \in \{0,1\}
            && \forall p \in \mathcal{P} \backslash\{\mathcal{P}'\}\label{con9}
    \end{align}
    Constraint

    Constraint (\ref{con1}) ensures all mandatory patients are assigned in the planning 
    horizon. Constraint (\ref{con2}) ensures that variables $x_{hdp}$ and $u_{hd}$ are not
    turned on simultaneously. Constraint (\ref{con3}) ensures that if a patient is assigned
    a hospital-day then that hospital-day is open. (\ref*{con4}) ensures that surgery time 
    of patients assigned to a hospital-day does not exceed the available surgery time in
    that hospital day. (\ref*{con5}) ensures an individuals surgery time does not exceed 
    the hospital-day's available time. (\ref*{con6}) ensures $y_{hd}$ gives a lower bound 
    on the number of operating rooms. (\ref*{con7}) ensures $y_{hd}$ does not exceed the
    number of operating rooms available on a hospital-day. Constraints (\ref*{con8}) -- (\ref*{con9}) simply restrict
    variables to binary.