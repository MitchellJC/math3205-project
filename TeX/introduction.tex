Our report is based upon and summarises Roshanaei, Luong, Aleman and Urbach~\cite{roshanaei2017propagating} which we will also refer to as \textit{the original paper} throughout. The paper solves the deterministic distributed  operating room scheduling (DORS) problem using Logic-Based Benders' Decomposition (LBBD) and a cut propagation method. This formulation has been extended upon in a stochastic setting\cite{guo}. We re-implemented the paper and extended the formulation with a new LBBD cut and a network model. We extended implementation by using lazy constraints in Gurobi. Our main goals were to replicate the results found in the original paper and investigate if there was potential improvement to be had by using lazy constraints instead of iteration.

% Structure scale and extendability of original models
The LBBD framework outlined in~\cite{roshanaei2017propagating} is structured as a location-allocation master problem with a bin-packing sub problem simplified by symmetric operating rooms. The results in the original paper show that the proposed LBBD models scale well for up to 160 patients.

% Overview of models in OG paper
The original paper contains many variants of models, they provide a pure MIP formulation, and an LBBD formulation with three different types of cuts. For each of these cuts they experiment with a version with and without propagation. For each of these they implement different cut generation schemes which determine how many sub problems to iterate over before re-solving the master problem. We choose some of their models to best achieve our goals. We choose to use the maximal cut generation scheme in all of our LBBD models, iterating over all sub-problems before re-solving the master problem.

% DORS Problem definition.
In the DORS problem, we seek to decide which hospital operating suites and their respective operating rooms to open. We also decide which patients to allocate to each hospital day. We make these decisions in such a way as to minimize the the total cost of opening facilities while also trying to maximize the reward for assigning critical patients\cite{roshanaei2017propagating}.