%%%%%%%%%%%%%%%%%%%%%%%%%%%%%%%%%%%%%%%%%
% Journal Article
% LaTeX Template
% Version 2.0 (February 7, 2023)
%
% This template originates from:
% https://www.LaTeXTemplates.com
%
% Author:
% Vel (vel@latextemplates.com)
%
% License:
% CC BY-NC-SA 4.0 (https://creativecommons.org/licenses/by-nc-sa/4.0/)
%
% NOTE: The bibliography needs to be compiled using the biber engine.
%
%%%%%%%%%%%%%%%%%%%%%%%%%%%%%%%%%%%%%%%%%

%----------------------------------------------------------------------------------------
%	PACKAGES AND OTHER DOCUMENT CONFIGURATIONS
%----------------------------------------------------------------------------------------

\documentclass[
	a4paper, % Paper size, use either a4paper or letterpaper
	10pt, % Default font size, can also use 11pt or 12pt, although this is not recommended
	unnumberedsections, % Comment to enable section numbering
	twoside, % Two side traditional mode where headers and footers change between odd and even pages, comment this option to make them fixed
]{LTJournalArticle}

\addbibresource{ref.bib} % BibLaTeX bibliography file

\runninghead{Shortened Running Article Title} % A shortened article title to appear in the running head, leave this command empty for no running head

\footertext{\textit{Journal of Biological Sampling} (2024) 12:533-684} % Text to appear in the footer, leave this command empty for no footer text

\setcounter{page}{1} % The page number of the first page, set this to a higher number if the article is to be part of an issue or larger work

%----------------------------------------------------------------------------------------
%	TITLE SECTION
%----------------------------------------------------------------------------------------

\title{Operating Room Scheduling} % Article title, use manual lines breaks (\\) to beautify the layout

% Authors are listed in a comma-separated list with superscript numbers indicating affiliations
% \thanks{} is used for any text that should be placed in a footnote on the first page, such as the corresponding author's email, journal acceptance dates, a copyright/license notice, keywords, etc
\author{%
	Ethan Merrick\textsuperscript{1}, Benjamin Solomon\textsuperscript{1} and Mitchell Clark\textsuperscript{1}\thanks{Corresponding author: \href{mailto:jane@smith.com}{jane@smith.com}\\ \textbf{Received:} October 20, 2023, \textbf{Published:} December 14, 2023}
}

% Affiliations are output in the \date{} command
\date{\footnotesize\textsuperscript{\textbf{1}}The University of Queensland}

% Full-width abstract
\renewcommand{\maketitlehookd}{%
	\begin{abstract}
		\noindent dope OR abstract
	\end{abstract}
}

% \documentclass[10pt, reqno]{amsart}
% % \documentclass[aps,twocolumn,secnumarabic,balancelastpage,amsmath,amssymb,nofootinbib,floatfix]{article}
\usepackage{setspace,tikz,xcolor,mathrsfs,listings,multicol, amsmath, amssymb}
\usepackage{rotating}

%\usepackage{subcaption}
%\usepackage{fullpage}
% \usepackage[all,cmtip]{xy}
% \usetikzlibrary{arrows,matrix}
% \usepackage[margin=1.25in]{geometry}

% % \onehalfspacing

% \usepackage[colorlinks=true, pdfstartview=FitV, linkcolor=blue, citecolor=blue, urlcolor=blue]{hyperref}

% % For breaking equations across multiple pages
% % \allowdisplaybreaks[1]
% \usepackage[colorinlistoftodos]{todonotes}
% \setlength\marginparwidth{1in}
% \newcommand{\ethan}[1]{\todo[size=\tiny,color=red!30]{#1 \\ \hfill --- Travis}}
% \newcommand{\Ethan}[1]{\todo[size=\tiny,inline,color=red!30]{#1
% 		\\ \hfill --- Travis}}
% \newcommand{\mitchell}[1]{\todo[size=\tiny,color=blue!30]{#1 \\ \hfill --- Mitchell}}
% \newcommand{\Mitchell}[1]{\todo[size=\tiny,inline,color=blue!30]{#1
% 		\\ \hfill --- Mitchell}}
% \newcommand{\benjamin}[1]{\todo[size=\tiny,color=green!30]{#1 \\ \hfill --- Benjamin}}
% \newcommand{\Benjamin}[1]{\todo[size=\tiny,inline,color=green!30]{#1
% 		\\ \hfill --- Benjamin}}



%%%%%%%%%%%%%%%%%%%%%%%%%%%%%%%%%%%%%%%%

\begin{document}
	% \title[]{The Proj}
	
	% \author[M.~Clark]{Mitchell Clark}
	% \author[B.~Solomon]{Benjamin Solomon}
    % \author[E.~Merrick]{Ethan Merrick}
	
	% \keywords{operations research}
	% \subjclass[2010]{}

	
	
	% \begin{abstract}
    % this is an abstract about some dope OR problem
	% change
	% \end{abstract}

	\maketitle
	% % \twocolumn
	% \setcounter{tocdepth}{1}
	% \tableofcontents

	\section{Introduction}
	%
	% This is just for clarity for what I have written
% Would be good if we could keep this somewhere in intro - Mitch
Our report is based upon~\cite{roshanaei2017propagating} which we will also refer to as \textit{the original paper} throughout. The paper solves the distributed operating room scheduling (DORS) problem using Logic-Based Benders' Decomposition (LBBD) and

	\section{Model Formulation}
	%main math from existing problem to define model
	 \subsection{Master Problem} 

The following is the LBBD formulation of the problem.
For a pure IP formulation refer to \cite{roshanaei2017propagating}. We will first outline
the formulation of the master problem. The master problem handles assignment of patients
to hospital-days.

\begin{table}[H]
% \captionsetup{font={Large,sf}}
\caption*{\bf{SETS}}
\resizebox{0.45\textwidth}{!}{
\begin{tabular}{ll}
    $\mathcal{P}$ &  Set of patients $p \in \mathcal{P}$\\
    $\mathcal{P}{'} $ & Set of mandatory patients, $\mathcal{P}{'} = \lbrace p|\rho_{p}(|\mathcal{D}|-\alpha_{p})\leq-\Gamma\rbrace$\\
    $\mathcal{H} $ & Set of hospitals, $h \in \mathcal{H}$\\
    $\mathcal{D} $ & Set of days in the planning horizon, $d \in \mathcal{D}$\\
    $\mathcal{R}_{h} $ & Set of ORs in each hospital's surgical suite, $r \in \mathcal{R}_{h}$\\
\end{tabular}
}

\end{table}

\begin{table}[H]
\caption*{\bf{DATA}}
\label{tab:MPdata}
\resizebox{0.45\textwidth}{!}{
\begin{tabular}{ll}
    $G_{hd}$ & Cost of opening the surgical suite in hosptial h on\\
    &  $\quad$ day d.\\
    $F_{hd} $ & Cost of opening and OR in hospital h on day d. \\
    $B_{hd} $ & Regular operating hours of each OR on day d. \\
                & $\quad$ in hospital h.\\
    $T_{hp} $ & Total booked time (preparation time + surgery time\\
               & $\quad$ + cleaning time) of patient p.\\
    $\rho_{p} $ & Health status score assigned to patient p.\\
    $\alpha_{p} $ & Number of days elapsed from the referal date of \\
    & $\quad$ patient p.\\
    $\kappa_{1} $ & Waiting cost for scheduled patients.\\
    $\kappa_{2} $ & Waiting cost for unscheduled patients.\\
    $\Gamma $ & Health status threshold above which patients have to\\
    & $\quad$ be operated.
\end{tabular}
}
\end{table}
\begin{table}[H]
\caption*{\bf{VARIABLES}}
\label{tab:MPvariables}
\resizebox{0.45\textwidth}{!}{
\begin{tabular}{ll}
    $x_{hdp} $ & 1 if patient p is assigned to hospital h on day d\\
    & $\quad$, 0 otherwise\\
    $u_{hd} $ & 1 if the surgical suite in hospital h is opened\\
    & $\quad$ on day d, 0 otherwise\\
    $y_{hd} $ & $\in \mathbb{Z}^+$,  lower bound on number of operating rooms \\
    & $\quad$ open in hospital h on day d \\
    $w_{p}$ & 1 if patient p is not scheduled this horizon,\\
    & $\quad$ 0 otherwise\\ 
\end{tabular}
}
\label{MP:variables}
\end{table}
\leavevmode
% \newline
% %setting up the master problem
\subsubsection*{Objective}

The objective function balances the minimisation of costs associated with opening hospitals
 and ORs and maximising the reward of assigning patients to surgeries. 
\begin{align}
\operatorname{minimize} \bigg( &\sum\limits_{h \in \mathcal{H}} \sum\limits_{d \in \mathcal{D}} G_{hd} U_{hd} 
 + \sum\limits_{h \in \mathcal{H}}\sum\limits_{d \in \mathcal{D}} F_{hd} y_{hd} \label{eq:MPobjective}
\\ &+ \sum\limits_{h \in \mathcal{H}} \sum\limits_{d \in \mathcal{D}} \sum\limits_{p \in \mathcal{P}}  
\kappa_{1} [\rho_{p} (d - \alpha_{p}) x_{hdp}]\notag
\\ &+ \sum\limits_{p \in \mathcal{P} \setminus \lbrace \mathcal{P}{'} \rbrace } \kappa_{2} [\rho_{p}( \mathcal{D} + 1 -\alpha_{p} ) w_{p}]
\bigg) \notag
\end{align}

\subsubsection*{Constraints} The constraints for the MP are formulated as follows.
\begin{align}
    \sum\limits_{h \in \mathcal{H}} \sum\limits_{d \in \mathcal{D}}x_{hdp} = 1 
        && \forall p \in \mathcal{P}' \label{MPcon1}\\
    \sum\limits_{h \in athcal{H}} \sum\limits_{d \in \mathcal{D}} x_{hdp} + w_p = 1
        && \forall p \in \mathcal{P} \backslash \{\mathcal{P}'\}\label{MPcon2}\\
    x_{hdp} \leq u_{hd}
        && \forall h \in \mathcal{H}, d \in \mathcal{D}, p \in \mathcal{P}\label{MPcon3}\\
    \sum\limits_{p \in \mathcal{P}}T_px_{hdp} \leq |\mathcal{R}_h|B_{hd}u_{hd}
        && \forall h \in \mathcal{H}, d \in \mathcal{D} \label{MPcon4}\\
    T_p\,x_{hdp} \leq B_{hd} 
        && \forall h \in \mathcal{H}, d \in \mathcal{D}, p \in \mathcal{P}\label{MPcon5}\\
     \frac{\sum_{p \in \mathcal{P}}T_px_{hdp}}{B_{hd}} \leq y_{hd}
        && \forall h \in \mathcal{H}, d\in \mathcal{D} \label{MPcon6}\\
        y_{hd} \leq |\mathcal{R}_h|
        && \forall h\in\mathcal{H}, d \in \mathcal{D} \label{MPcon7}\\
        u_{hd}, x_{hdp} \in \{ 0,1\}
        && \forall h\in \mathcal{H}, d \in \mathcal{D}, p \in \mathcal{P}\label{MPcon8}\\
        w_p \in \{0,1\}
        && \forall p \in \mathcal{P} \backslash\{\mathcal{P}'\}\label{MPcon9}
\end{align}
Constraint (\ref{MPcon1}) ensures all mandatory patients are assigned in the planning 
horizon. Constraint (\ref{MPcon2}) ensures that variables $x_{hdp}$ and $u_{hd}$ are not
turned on simultaneously. Constraint (\ref{MPcon3}) ensures that if a patient is assigned
a hospital-day then that hospital-day is open. (\ref*{MPcon4}) ensures that surgery time 
of patients assigned to a hospital-day does not exceed the available surgery time in
that hospital day. (\ref*{MPcon5}) ensures an individuals surgery time does not exceed 
the hospital-day's available time. (\ref*{MPcon6}) ensures $y_{hd}$ gives a lower bound 
on the number of operating rooms. (\ref*{MPcon7}) ensures $y_{hd}$ does not exceed the
number of operating rooms available on a hospital-day. Constraints (\ref*{MPcon8}) -- (\ref*{MPcon9}) simply restrict
variables to binary.

\subsection{Subproblems}
Given a solution to the master problem $(\widehat{Y}^{(i)}_{hd}, \widehat{\mathcal{P}}^{(i)}_{hd})$ the 
subproblem minimises the number of ORs to open for a given hospital day. Each subproblem is
formulated as follows.

\begin{table}[H]
    \caption*{\bf{ADDITIONAL VARIABLES}}
    \resizebox{0.45\textwidth}{!}{
        \begin{tabular}{ll}
            $y_r$ & $\in \mathbb{Z}^+$, number of open operating rooms. \\
            $x_{pr}$ & 1 if patient p is assigned to operating room r,\\
            &   $\quad$ 0 otherwise.
        \end{tabular}
    }
\end{table}
\begin{align}
    \operatorname*{minimise} \quad \overline{Y}_{hd} = \sum\limits_{r \in \mathcal{R}_h}y_r
\end{align}

With constraints given as follows:
\begin{align}
    \sum\limits_{r\in\mathcal{R}_h}x_{pr} = 1 && \forall p \in \widehat{\mathcal{P}}^{(i)}_{hd}\label{SPcon1}\\
    \sum\limits_{p \in \widehat{\mathcal{P}}^{(i)}_{hd}} T_px_{pr} \leq B_{hd}y_r 
    && \forall r \in \mathcal{R}_h \label{SPcon2}\\
    x_{pr} \leq y_r && \forall p \in \widehat{\mathcal{P}}^{(i)}_{hd}, r \in \mathcal{R}_h \label{SPcon3}\\
    y_r \leq y_{r-1} && \forall r \in \mathcal{R}_h \backslash \{1\}\label{SPcon4}\\
    x_{pr},\,y_r\in\{0,1\} && \forall p \in \widehat{\mathcal{P}}^{(i)}_{hd}, r \in \mathcal{R}_h\label{SPcon5}
\end{align}
Constraint (\ref*{SPcon1}) ensures that each patient is assigned to only one operating room.
Constraint (\ref*{SPcon2}) ensures that no OR is overcapacitated. Constraint (\ref*{SPcon3})
ensures that patients are assigned to open ORs. Constraint (\ref*{SPcon4}) breaks symmtry
among ORs.

\subsection{Benders Cuts}
There are multiple forms of benders cuts outlined in \cite{roshanaei2017propagating}. We will discuss
pertinent forms based on their performance as discussed in \cite{roshanaei2017propagating}. These are 
the LBBD1 and LBBD2 benders cuts. In order to describe these cut types, we will first discuss the 
first-fit decreasing heuristic algorithm (FFD) as this is used to determine feasibility of SPs. 

\subsubsection{First-fit decreasing heuristic algorithm}
Since the SP packing problem can be difficult to solve, we can first find a feasible solution 
($\overline{F}^{(i)}_{hd}$) using a FFD heuristic. This process is faster than other techniques
such as integer and constraint programming. Moreover, we have the following relationship 
between the FFD, MP and SP solutions;
\begin{equation}
    \tilde{Y}^{(i)}_{hd} \leq \overline{Y}^{(i)}_{hd} \leq \overline{F}^{(i)}_{hd}
\end{equation}
We can use the FFD solution $\left(\overline{F}^{(i)}_{hd}\right)$ to find an optimal SP solution 
$\left(\overline{Y}^{(i)_{hd}}\right)$ without explicitly solving the SP. Moreover, if 
$\overline{F}^{(i)}_{hd} \neq \tilde{Y}^{(i)}_{hd}$ then when solving the SP we can use 
$\operatorname{min}\{\overline{F}^{(i)}_{hd},\, |\mathcal{R}_h|\}$ as an upper bound. 

\subsubsection{LBBD1}
LBBD1 \cite{roshanaei2017propagating} utilises both feasibility and optimality cuts to 
correct the MP to find a solution. If the SP is infeasible the following "no good" cut 
is added to the MP which requires at least one patient be removed from 
$\tilde{\mathcal{P}}^{(i)}_{hd}$.
\begin{align}
    \sum\limits_{p \in \tilde{\mathcal{P}}^{(i)}_{hd}}(1-x_{hdp})\geq 1 
        && \forall (h,d) \in \mathcal{U}_{hd}^{(i)}
\end{align}
Where $\mathcal{U}_{hd}^{(i)}$ is the set of infeasible SPs at this stage of solving the MP.
If the SP is optimal, that is $\tilde{Y}^{(i)}_{hd} = \overline{Y}^{(i)}_{hd}$, no cuts
are required. However, if the this is not the case, an optimality cut must be added.

\subsubsection{LBBD2}

\subsection{Network Problem}
We also give the formulation of the problem as a network. The problem contains the sets
and data of the master problem previously outlined with the addition of the following 
sets and data. 
\begin{table}[H]
    % \captionsetup{font={Large,sf}}
    \caption*{\bf{SETS}}
    \resizebox{0.4\textwidth}{!}{
    \begin{tabular}{ll}
        $\mathcal{N}$ &  Set of nodes $n \in \mathcal{N}$\\
        $\mathcal{N^\prime}$ & Set of nodes $n \in \mathcal{N}$ \\
        & $\operatorname{s.t.} \operatorname{minDur} \leq n[time] \leq B_{n[time],n[day]} - \operatorname{minDur} $ \\
        $\mathcal{A}$ &  Set of arcs $a \in \mathcal{A}$\\
    \end{tabular}
    }
    \end{table}
    \begin{table}[H]
        % \captionsetup{font={Large,sf}}
        \caption*{\bf{DATA}}
        \resizebox{0.35\textwidth}{!}{
        \begin{tabular}{ll}
            $t_n$ & Arcs that enter node $n \in \mathcal{N}$ \\
            $f_n$ & Arcs that leave node $n \in \mathcal{N}$\\
            $\operatorname{minDur}$ & minimum surgery duration
        \end{tabular}
        }
        \end{table}
    The variables are also shared with the master problem with the addition of the following variable. 
\begin{table}[H]
    \caption*{\bf{VARIABLES}}
    \label{tab:NETvariables}
    \resizebox{0.4\textwidth}{!}{
    \begin{tabular}{ll}
        $z_{a}$ & 1 if arc a is turned on, 0 otherwise.
    \end{tabular}
    }
    \end{table}

    \subsubsection*{Objective}
    The objective function is identical to master problem objective given in \ref{eq:MPobjective}.
    \begin{align}
    \operatorname{minimize} \bigg( &\sum\limits_{h \in \mathcal{H}} \sum\limits_{d \in \mathcal{D}} G_{hd} U_{hd} 
     + \sum\limits_{h \in \mathcal{H}}\sum\limits_{d \in \mathcal{D}} F_{hd} y_{hd}
    \\ &+ \sum\limits_{h \in \mathcal{H}} \sum\limits_{d \in \mathcal{D}} \sum\limits_{p \in \mathcal{P}}  \label{MP:objective}
    \kappa_{1} [\rho_{p} (d - \alpha_{p}) x_{hdp}]
    \\ &+ \sum\limits_{p \in \mathcal{P} \setminus \lbrace \mathcal{P}{'} \rbrace } \kappa_{2} [\rho_{p}( \mathcal{D} + 1 -\alpha_{p} ) w_{p}]
    \bigg) \notag
    \end{align}
    
    \subsubsection*{Constraints} The constraints for the network formulation are given as follows,
    \begin{align}
        \sum\limits_{a \in f_n} z_a = \sum\limits_{a \in t_n} z_a 
            && \forall n \in \mathcal{N}\label{NETcon:room_flow}\\
        \sum\limits_{a \in \mathcal{A}}
            &&\\
        \sum\limits_{h \in \mathcal{H}} \sum\limits_{d \in \mathcal{D}}x_{hdp} = 1 
            && \forall p \in \mathcal{P}' \label{con1}\\
        \sum\limits_{h \in athcal{H}} \sum\limits_{d \in \mathcal{D}} x_{hdp} + w_p = 1
            && \forall p \in \mathcal{P} \backslash \{\mathcal{P}'\}\label{con2}\\
        x_{hdp} \leq u_{hd}
            && \forall h \in \mathcal{H}, d \in \mathcal{D}, p \in \mathcal{P}\label{con3}\\
        \sum\limits_{p \in \mathcal{P}}T_px_{hdp} \leq |\mathcal{R}_h|B_{hd}u_{hd}
            && \forall h \in \mathcal{H}, d \in \mathcal{D} \label{con4}\\
        T_p\,x_{hdp} \leq B_{hd} 
            && \forall h \in \mathcal{H}, d \in \mathcal{D}, p \in \mathcal{P}\label{con5}\\
         \frac{\sum_{p \in \mathcal{P}}T_px_{hdp}}{B_{hd}} \leq y_{hd}
            && \forall h \in \mathcal{H}, d\in \mathcal{D} \label{con6}\\
            y_{hd} \leq |\mathcal{R}_h|
            && \forall h\in\mathcal{H}, d \in \mathcal{D} \label{con7}\\
            u_{hd}, x_{hdp} \in \{ 0,1\}
            && \forall h\in \mathcal{H}, d \in \mathcal{D}, p \in \mathcal{P}\label{con8}\\
            w_p \in \{0,1\}
            && \forall p \in \mathcal{P} \backslash\{\mathcal{P}'\}\label{con9}
    \end{align}

    Constraint (\ref{con1}) ensures all mandatory patients are assigned in the planning 
    horizon. Constraint (\ref{con2}) ensures that variables $x_{hdp}$ and $u_{hd}$ are not
    turned on simultaneously. Constraint (\ref{con3}) ensures that if a patient is assigned
    a hospital-day then that hospital-day is open. (\ref*{con4}) ensures that surgery time 
    of patients assigned to a hospital-day does not exceed the available surgery time in
    that hospital day. (\ref*{con5}) ensures an individuals surgery time does not exceed 
    the hospital-day's available time. (\ref*{con6}) ensures $y_{hd}$ gives a lower bound 
    on the number of operating rooms. (\ref*{con7}) ensures $y_{hd}$ does not exceed the
    number of operating rooms available on a hospital-day. Constraints (\ref*{con8}) -- (\ref*{con9}) simply restrict
    variables to binary.

	\section{Data}
	% talk about data generation and accuracy
	We generate our own data based on the parameters provided in the original paper. The original instances used were not available. Using examples given by Guo\cite{guo} we round surgery times to integer values. We generated data according to the distributions in Table~\ref{tab:dataDist}.

\begin{table}[H]
    \centering
    \caption{Distributions used for data generation.}\label{tab:dataDist}
    \begin{tabular}{ll} \toprule
        Data & Distribution \\\midrule
        $\kappa_1$ & 50 dollars \\
        $\kappa_2$ & 5 dollars \\
        $\Gamma$ & 500 \\
        $\rho_p$ & Discrete uniform distribution [1, 5] \\
        $B_{hd}$ & Discrete uniform distribution [420, 480] minutes  \\
        & \quad in 15-minute intervals \\
        $\alpha_p$ & Discrete uniform distribution [60, 120] days. \\
        $F_{hd}$ & Discrete uniform distribution [4000, 6000] dollars \\
        $G_{hd}$ & Discrete uniform distribution [1500, 2500] dollars \\
        $T_p$ & 
            Truncated normal distribution [45, 480] minutes, \\
        & \quad $\mu=160$, $\sigma=40$
         \\
        \bottomrule
    \end{tabular}
\end{table}

	\section{Implementation}
	%
	% Specs and general overview
Models were run on a Windows 11 computer with a AMD Ryzen 5 5625U 2.3GHz processor with 16 GB of RAM\@. Gurobi version 10.0.1 was used with Python version 3.9.16. We implemented two different models for LBBD, one following\cite{roshanaei2017propagating} which solves the master problem iteratively in a loop, rebuilding the master problem branch-and-bound tree on every iteration. The other model leveraged lazy constraints in Gurobi to perform a branch and check routine\cite{LBBDBible}, building the branch-and-bound tree once while applying Benders' cuts as lazy constraints at each new incumbent solution. The pure MIP and network model were solved using standard Gurobi routines. 



	\section{Results}
	%
	% Runtime specifications
Models were run on a Windows 11 computer with a AMD Ryzen 5 5625U 2.3GHz processor with 16 GB of RAM\@. Gurobi version 10.0.1 was used with Python version 3.9.16. 

We aimed to compare 7 models: The pure MIP, the network model, an iterative LBBD1 (iLBBD1), LBBD1 using lazy constraints (cLBBD1), iterative LBBD2 with propagation (iLBBD2p), LBBD2 with lazy constraints and propagation (cLBBD2p) and LBBD4 with lazy constraints and propagation (cLBBD4). The pure MIP was used as a baseline.\ iLBBD1 and cLBBD1  were chosen to be used as baseline LBBD models to compare lazy constraints to iteration. LBBD2 with propagation was chosen as it was one of the best performers in th original papers, with the idea to use it as the main comparison to the network model and our new cut. We chose to collect results for LBBD4 only using propagation and lazy constraints as we believed this would give us the best results while adhering to the project time constraints.

We generate 5 seeded instances of data as done in~\cite{roshanaei2017propagating} we only run each model over each instance once unlike~\cite{roshanaei2017propagating}. Unlike in the original paper, where models were ran with a time limit of 7200 seconds, we set our time limit to 900 seconds due to project time constraints. At a time limit of 7200 seconds the worst case time to run all models over all 5 instances was 70 hours for only one patient size. For this same reason we only ran models with 5 available operating rooms, instead of testing 5 and 3. We chose to use 5 operating rooms as this should lead to harder sub problems and relatively easier master problems\cite{roshanaei2017propagating}. We run the models on 5 different instances of data for each number of patients. There was uncertainty as to whether the models were run to true optimality or to a relative MIP gap of $1\%$ as was done by~\cite{guo}, so we report results for both scenarios. We follow~\cite{roshanaei2017propagating} by defining the best performing model to be the most robust, that is, the one able to solve the model to optimality within the given time constraints.

\begin{table*}
    \centering
    \caption{Average time (seconds) until solved to optimality over 5 instances. The number of instances not solved to optimality are superscripted. Non-solved instances are not included in average.}
    \begin{tabular}{rrrrrrrr} \toprule
        $|\mathcal{P}|$ & Pure MIP & Network & iLBBD1 & cLBBD1 & iLBBD2p & cLBBD2p & cLBBD4p \\ \midrule
        20              & 16.06 &         & 1.509 &  0.8829 & 1.431 & 0.8800 & 0.7890 \\
        40              & $179.290365^{(2)}$ &    &  $4.428572$ & $1.911088^{(4)}$ & $4.503378^{(1)}$ & $1.910761^{(4)}$ & $1.959761^{(4)}$ \\
        60 & $30.804891^{(4)}$ &  & $24.4599^{(1)}$ & $10.726104^{(4)}$ & $34.606863^{(2)}$ & $21.455653^{(4)}$ & $25.795459^{(4)}$ \\
        \bottomrule
    \end{tabular}
\end{table*}


\begin{table*}
    \centering
    \caption{Average relative MIP gap over 5 instances after trying to solve to optimality.}
    \begin{tabular}{rrrrrrrr} \toprule
        $|\mathcal{P}|$ & Pure MIP & Network & iLBBD1 & cLBBD1 & iLBBD2p & cLBBD2p & cLBBD4p \\ \midrule
        20              & 0 &         & 0 &  0 & 0 & 0 & 0 \\
        40              & $4.220 \times 10^{-4}$ &   & 0.0 & $3.532 \times 10^{-3}$ & $9.500 \times 10^{-5}$ & $4.765 \times 10^{-3}$ & $4.948 \times 10^{-3}$ \\
        60 & $1.827 \times 10^{-3}$ &  & $8.000 \times 10^{-6}$ & $2.540 \times 10^{-3}$ & $1.700 \times 10^{-5}$ & $3.513 \times 10^{-3}$ & $3.273 \times 10^{-3}$ \\
        \bottomrule
    \end{tabular}
\end{table*}


	\section{Discussion}
	%
	\input{improvements.tex}
	
	\section{Conclusion}
	%
	% Conclusion
We reimplemented a pure MIP and Benders' Decomposition models from~\cite{roshanaei2017propagating}. Models were extended by constructing a new Benders' Cut LBBD4 and by trialling a network model. Implementation was extended by using lazy constraints in Gurobi. Implementations succesfully generated consistent optimal solutions across all models. Implementations generated feasible solutions from observation. Models were able to solve to optimality for only a small patient set size of 20 but were able to scale up to 80 patients when solving to a 1\% gap. We were not able to replicate the efficiency of Benders' Decomposition from the original paper, this was believed to be caused by some sub-optimal implementation used or the difference data generation.  

From our results, the pure MIP model performed the best overall, being the most robust when solving to optimality and being the fastest to solve across all sizes of patient sets when solving to a 1\% gap. The Network model was the least robust when solving to optimality and the slowest when solving to a 1\% gap. Overall, callback implementations performed slightly better than iterative variants.

% Future improvements/work.
Discretizing time further into periods may remedy some of the scaling issues found with all models when solving to optimality, this would sacrifice accuracy for speed, although this can already be done by simply using a larger gap. 

Larger number of patients could be tested to find the limits of the models when solving to a 1\% gap as it was seen that all models were still quite robust up to 80 patients. 

Further algorithmic optimization could be trialled. Caching of sub problems was used to check for identical sub problems but exploratory tests using a more sophisticated caching protocal showed promise, project time constraints and long run-times constrained the collection of more results. The more sophisticated caching methods depend on looking at the sorted set of patients for a given solved sub problem. And the sorted set of patients for the current sub problem. If patient-wise times for the current sub problem are less than or equal to those of a feasible cached sub problem, we know a feasible solution to the current sub problem. Similarly, if the patient-wise time for the current sub problem are greater than or equal to those of an infeasible cached sub problem, we know the current sub problem must be infeasible. The results for this may show great improvement, especially for lazy constraint implementations where similar sub problems are visited often.




	%=====================================================================
	\printbibliography % Output the bibliography
\end{document}